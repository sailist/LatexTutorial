\begin{document}
  % 摘要
  \begin{abstract}
    This is a test abstract
    % 关键字
    \begin{keywords}
    keyword1; keyword2
    \end{keywords}
  \end{abstract}

\maketitle

%%段落
\section{Introduction}

%% 子段落
\subsection{Other Assumptions}

% 加粗颜色(两个叠加)
\textbf{\textcolor[rgb]{0.98,0.00,0.00}{Input matlab source:}}
% 字体颜色
\textcolor[rgb]{0.98,0.00,0.00}{Input matlab source:}
%% 点列表
\begin{itemize}
  %%加粗
  \item \textbf{Applies widely}\\
  This  system can be used for many types of airplanes
  \item maximizes the outgoing velocity of the ball.
\end{itemize}

%% 着重强调
\emph{center of percussion} [Brody 1986]

%% 定理
\begin{Theorem} \label{thm:latex}
this is a Theorem
\end{Theorem}

%% 引理
\begin{Lemma} \label{lem:tex}
this is a Lemma
\end{Lemma}

%% 证明
\begin{proof}
this is a proof
\end{proof}

%% 公式 
%% https://www.codecogs.com/latex/eqneditor.php
%% http://www.hostmath.com/
\begin{equation}
Q(x,y;r,p,e)
\end{equation}
  

%% 表格,在线生成表格网站:http://www.tablesgenerator.com/
%% 或http://truben.no/latex/table/
\begin{table}[]
  \begin{tabular}{lllll}
   &  &  &  &  \\
   &  &  &  &  \\
   &  &  &  &  \\
   &  &  &  & 
  \end{tabular}
\end{table}

%% 图片
\begin{figure}[htb]
  \small
  \centering
  \includegraphics[width=12cm]{graph-model.eps}
  \caption{this input the figure name} \label{fig:temp}
\end{figure}
 
%% 多张图片
\begin{figure}[htbp]
  \centering
  \subfigure[pic1.]{
  \includegraphics[width=5.5cm]{mcmthesis-aaa.eps}
  }
  \quad
  \subfigure[pic2.]{
  \includegraphics[width=5.5cm]{mcmthesis-aaa.eps}
  } 
  \quad
  \subfigure[pic3.]{
  \includegraphics[width=5.5cm]{mcmthesis-aaa.eps}
  }
  \quad
  \subfigure[pic4.]{
  \includegraphics[width=5.5cm]{mcmthesis-aaa.eps}
  }
  \caption{ pics}
\end{figure}


this is a ref to figure-\ref{fig:temp}

\[
  \begin{pmatrix}{*{20}c}
  {a_{11} } & {a_{12} } & {a_{13} }  \\
  {a_{21} } & {a_{22} } & {a_{23} }  \\
  {a_{31} } & {a_{32} } & {a_{33} }  \\
  \end{pmatrix}
  = \frac{{Opposite}}{{Hypotenuse}}\cos ^{ - 1} \theta \arcsin \theta
\]
 
\[
  p_{j}=\begin{cases} 0,&\text{if $j$ is odd}\\
  r!\,(-1)^{j/2},&\text{if $j$ is even}
  \end{cases}
\]


\[
  \arcsin \theta  =
  \mathop{{\int\!\!\!\!\!\int\!\!\!\!\!\int}\mkern-31.2mu
  \bigodot}\limits_\varphi
  {\mathop {\lim }\limits_{x \to \infty } \frac{{n!}}{{r!\left( {n - r}
  \right)!}}} \eqno (1)
\]

  % %% 引用页
  % \begin{thebibliography}{99}
  %   % Greff2015LSTM
  %   \bibitem{b1} D.~E. KNUTH   The book  the American
  %     Mathematical Society and Addison-Wesley
  %     Publishing Company , 1984-1986.
  %   \bibitem{b2}Lamport, Leslie, : `` A Document Preparation System '',Addison-Wesley Publishing Company, 1986.
  %   \bibitem[Sza]{bt}Lamport, Leslie, : `` A Document Preparation System '',Addison-Wesley Publishing Company, 1986.


  %   \bibitem{b3}\url{http://www.latexstudio.net/}
  % \end{thebibliography}

  %% 使用引用
  this is the usage of cite: \cite{Greff2015LSTM} 

  %% 添加引用(固定格式)
  \bibliographystyle{abbrv}
  \bibliography{bibfile} 

  %% 附录页
  \begin{appendices}
    \section{First appendix}
    \textbf{\textcolor[rgb]{0.98,0.00,0.00}{Input matlab source:}}
    %% 附加代码,支持C,C++,Java,Matlab,Mathematica
    % \lstinputlisting[language=Matlab]{./code/mcmthesis-matlab1.m}
    \section{Second appendix}

    some more text \textcolor[rgb]{0.98,0.00,0.00}{\textbf{Input C++ source:}}
    % \lstinputlisting[language=C++]{./code/mcmthesis-sudoku.cpp}
  \end{appendices}
