\documentclass{beamer}
    \usepackage[UTF8]{ctexcap} %用于支持中文
    % \usepackage[utf8]{inputenc} % 这个包没法支持中文

    \usetheme{CambridgeUS} %这里来定义主题,如果要更多的主题,可以访问 http://deic.uab.es/~iblanes/beamer_gallery/index_by_theme.html

    \begin{document}
        \begin{frame}
            \title[Android的介绍]{Latex beamer介绍}
            \subtitle {--从文档自己展示起}
            \author{Sailist}
            \date{\today}
            \titlepage %这里必须写上让上面的内容显示出来
        \end{frame}
        
        \begin{frame}{目录}
            该页定义了目录

            \tableofcontents
        \end{frame}

        \begin{frame}{标题的第一种出现方式}
        \end{frame}

        \begin{frame}
            \frametitle{标题的另一种写法}
        \end{frame}
        
        \begin{frame}
            这里可以写一些内容,是正常的文本。\\
            Here can write some content.
        \end{frame} 
        
        \begin{frame}
            下面定义的是列表,和Latex一样。\\
            Here define the itemize , the grammer is the same as other latex code.
            \begin{itemize}
                \item item 1
                \item item 2
            \end{itemize}
        \end{frame} 
        
        \begin{frame}
            使用不同的主题,可以有不同的环境色.\\
            Use different theme can change the style.\\
            所有的可供选择的主题所在的链接:\\
           % http://deic.uab.es/~iblanes/beamer_gallery/index_by_theme.html
        \end{frame}

        \section{这里可以定义标题Here define the section.}
        \begin{frame}
            
            嘿,看上面!\\
            Hey,look the top line.

        \end{frame}


        \subsection{这里定义了子标题Here define the subsection} %the beamer may not support subsubsection.
        \begin{frame}
            
            继续看。\\
            continue look.
        \end{frame}
        \section{动画入门 start animate}
        \begin{frame}

            只支持逐步切换,这是第一张。\\
            This is the first step.\\
            \pause
            只支持逐步切换,这是第二张。\\
            This is the second stop.
            
        \end{frame}
         

        \begin{frame}
            \frame[plain]
            这一页定义了plain方式。
            This slide define the plain style.
        \end{frame}

    \end{document}